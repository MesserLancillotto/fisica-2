
\documentclass{article}
\title{Risoluzione tema esame di fisica 2 del 3 settembte 2020}
\author{Lancillotto dal lago}
\date{\today}

\usepackage[utf8]{inputenc}
\usepackage{enumerate}
\usepackage{amsmath}
\usepackage{mathtools}
\usepackage{graphicx}

\usepackage[right=1.5cm, left=1.5cm]{geometry}

\graphicspath{ {./} }

\begin{document}
\pagenumbering{gobble}
\maketitle

\begin{enumerate}
    
    % 1
    \item Due sferette di massa $m_1 = m_2 = m = 20g$ e caria $q_1 = q$ e $q_2 = 2q$, sono appese a
    due fili di lunghezza $l = 120cm$, che formano due angoli $\theta_1$ e $\theta2$ molto piccoli
    con la verticale. Calcolare:
    
    \begin{enumerate}
        % 1.1
        \item Il rapporto $\theta_1 / \theta_2$
        \newline 
        \textbf{Risoluzione}:
        % 1.2
        \item Se la distanza fra le due sferette all’equilibrio è $r = 10cm$, calcolare il valore di $q$
        \newline 
        \textbf{Risoluzione}:

    \end{enumerate}

    % 2
    \item Due cariche $q_1 = -2 \cdot 10^{-8} C$ e $q_2 = 5 \cdot 10^{-8} C$ sono poste lungo una diagonale di un
    rettangolo di lati $a = 30cm$ e $b = 20cm$. Calcolare:

    \begin{enumerate}
        
        % 2.1
        \item Il vettore campo elettrico nei due vertici dove si trova la carica q_3
        \newline 
        \textbf{Risoluzione}:

        % 2.2
        \item La forza che agisce sulla carica in entrambi i vertici
        \newline 
        \textbf{Risoluzione}:

        % 2.3
        \item Il lavoro compiuto dalle forze elettrostatiche per spostare una carica $q_3$
        = $0.5 \cdot 10{-9}$ da un vertice a quello opposto
        \newline 
        \textbf{Risoluzione}:

    \end{enumerate}

    % 3
    \item Una bobina composta da N=100 spire di raggio R=10cm, giace sul piano xy ed è percorsa dalla corrente
    i=8A, in senso antiorario. Essa è sottoposta all’azione di un campo magnetico B=0.6ux -0.4uy+0.2uz T.
    Calcolare:

    \begin{enumerate}
        
        % 3.1
        \item Il momento magnetico m della bobina
        \newline 
        \textbf{Risoluzione}:

        % 3.2
        \item Il momento meccanico M che agisce sulla spira
        \newline 
        \textbf{Risoluzione}:
        
        % 3.3
        \item L’energia potenziale magnetica U_m 
        \newline 
        \textbf{Risoluzione}:

    \end{enumerate}

    % 4
    \item Due lenti convergenti, di distanze focali $f_1 = 30 cm$ e $f_2 = 15 cm$, sono
    disposte in aria, perpendicolarmente all’asse $x$, ad una distanza di $d =
    20 cm$. Anche i fuochi delle lenti sono situati sull’asse $x$. Un oggetto
    viene posto ad una distanza $p = 10 cm$ dalla prima lente.

    \begin{enumerate}
        
        % 4.1
        \item Determinare l’ingrandimento complessivo e la posizione
        dell’immagine in uscita alla seconda lente
        \newline 
        \textbf{Risoluzione}:

        % 4.2
        \item Tracciare graficamente il percorso ottico e la posizione finale dell’oggetto
        \newline 
        \textbf{Risoluzione}:

    \end{enumerate}

\end{document}