
\documentclass{article}
\title{Risoluzione tema esame di fisica 2 del 3 settembte 2020}
\author{Lancillotto dal lago}
\date{\today}

\usepackage[utf8]{inputenc}
\usepackage{enumerate}
\usepackage{amsmath}
\usepackage{mathtools}
\usepackage{graphicx}

\usepackage[right=1.5cm, left=1.5cm]{geometry}

\graphicspath{ {./} }

\begin{document}
\pagenumbering{gobble}
\maketitle

\begin{enumerate}
    
    % 1
    \item Due sferette di massa $m_1 = m_2 = m = 20g$ e caria $q_1 = q$ e $q_2 = 2q$, sono appese a
    due fili di lunghezza $l = 120cm$, che formano due angoli $\theta_1$ e $\theta2$ molto piccoli
    con la verticale. Calcolare:
    
    \begin{enumerate}
        % 1.1
        \item Il rapporto $\theta_1 / \theta_2$
        \newline 
        \textbf{Risoluzione}: Si inizia scrivendo le componenti delle forze che agiscono sulle sferette: quella gravitazionale che agisce in verticale 
        e quella elettrostatica che agisce in orizzontale, ottenendo così 
        \begin{align*}
            F_g & = mg \\
            F_e & = \frac{1}{4 \pi \epsilon_0} \frac{2q \cdot q}{d^2}
        \end{align*}
        Si nota che le forze sono uguali in modulo ed opposte in verso, dunque la forza elettrostatica è uguale, per poi notare che anche la componente verticale 
        è identica essento identiche le masse. Si potrebbero poi ricordare le formule che permettono di calcolare il rapporto, oppure più semplicemente si nota come le 
        due cariche siano simmetriche rispetto la verticale e che quindi $\theta_1 = \theta_2 \Rightarrow \theta_1 / \theta_2 = 1$ 
        % 1.2
        \item Se la distanza fra le due sferette all’equilibrio è $r = 10cm$, calcolare il valore di $q$
        \newline 
        \textbf{Risoluzione}: Tratto tutto come un problema geometrico, avendo le proiezioni dei fili e le forze sia verticali che orizzontali, calcolo il rapporto 
        $F_e : h = F_g : \frac{r}{2}$ (con $h$ distanza tra la retta passante per le sferette ed il punto in cui sono appese, di valore 
        $h = \sqrt{l^2 - \left( \frac{r}{2}\right)^2}$):
        \begin{align*}
            \frac{2q^2}{4 \pi \epsilon_0 r^2} : \frac{r}{2} = mg : \sqrt{l^2 - \left( \frac{r}{2}\right)^2} \\
            \frac{2q^2}{4 \pi \epsilon_0 r^2} = 2 \cdot \frac{mg \cdot \sqrt{l^2 - \left( \frac{r}{2}\right)^2}}{r} \\ 
            q = 4 \pi \epsilon_0 r \cdot mg \cdot \sqrt{l^2 - \left( \frac{r}{2}\right)^2}
        \end{align*}
        \textbf{Domanda}: Cos'è quel $2$ all'inizio della seconda equazione, e che fine ha fatto il denominatore nella terza?
        \newline
        Il $2$ viene da $\frac{r}{2}$ che essendo al denominatore salta su, salvo poi semplificarsi nella terza con il $2$ della $q$ ed $r$ con $r^2$ al numeratore.
 
    \end{enumerate}

    % 2
    \item Due cariche $q_1 = -2 \cdot 10^{-8} C$ e $q_2 = 5 \cdot 10^{-8} C$ sono poste lungo una diagonale di un
    rettangolo di lati $a = 30cm$ e $b = 20cm$. Calcolare:

    \begin{enumerate}
        
        % 2.1
        \item Il vettore campo elettrico nei due vertici dove si trova la carica $q_3$
        \newline 
        \textbf{Risoluzione}: Si applica la formula del campo ricavandone le componenti, ricordando che se la forza è positiva allora è repulsiva, altrimenti è attrattiva.
        
        \begin{align*}
            E_1x & = \frac{1}{4 \pi \epsilon_0} \frac{q_1}{a^2} \\
            E_1y & = \frac{1}{4 \pi \epsilon_0} \frac{q_2}{b^2} \\
            E_2x & = \frac{1}{4 \pi \epsilon_0} \frac{q_2}{a^2} \\
            E_2x & = \frac{1}{4 \pi \epsilon_0} \frac{q_1}{b^2}
        \end{align*}

        Da qui avrò che il vettore campo negli angoli del quadrilaterò sarà 
        \begin{align*}
            E_1 & = \vec{u_x} + \vec{u_y} \\
            |E_1| & = RISULTATO \\
            E_2 & = \vec{u_x} + \vec{u_y} \\
            |E_2| & = RISULTATO
        \end{align*}



        % 2.2
        \item La forza che agisce sulla carica in entrambi i vertici
        \newline 
        \textbf{Risoluzione}: La ottengo moltiplicando il campo $E$ per la carica $q_3$ nel punto
        \begin{align*}
            F_1 & = q_3E_1 = RISULTATO \\
            F_2 & = q_3E_2 = RISULTATO
        \end{align*}

        % 2.3
        \item Il lavoro compiuto dalle forze elettrostatiche per spostare una carica $q_3$
        = $0.5 \cdot 10^{-9}$ da un vertice a quello opposto
        \newline 
        \textbf{Risoluzione}: Il lavoro è definito $W = -\Delta U = -\Delta V$, da qui
        \begin{align*}
            V_1 & = \frac{1}{4 \pi \epsilon_0}\frac{q_1q_3}{a} + \frac{1}{4 \pi \epsilon_0}\frac{q_2q_3}{b} \\ 
                & = \frac{q_3}{4 \pi \epsilon_0} \left( \frac{q_1}{a} + \frac{q_2}{b} \right) \\ 
            V_2 & = \frac{1}{4 \pi \epsilon_0}\frac{q_1q_3}{b} + \frac{1}{4 \pi \epsilon_0}\frac{q_2q_3}{a} \\ 
                & = \frac{q_3}{4 \pi \epsilon_0} \left( \frac{q_1}{b} + \frac{q_2}{a} \right) \\
            W   & = V_2 - V_1 = RISULTATO
        \end{align*} 


    \end{enumerate}

    % 3
    \item Una bobina composta da N=100 spire di raggio R=10cm, giace sul piano xy ed è percorsa dalla corrente
    i=8A, in senso antiorario. Essa è sottoposta all’azione di un campo magnetico B=0.6ux -0.4uy+0.2uz T.
    Calcolare:

    \begin{enumerate}
        
        % 3.1
        \item Il momento magnetico m della bobina
        \newline 
        \textbf{Risoluzione}:

        % 3.2
        \item Il momento meccanico M che agisce sulla spira
        \newline 
        \textbf{Risoluzione}:
        
        % 3.3
        \item L’energia potenziale magnetica $U_m$
        \newline 
        \textbf{Risoluzione}:

    \end{enumerate}

    % 4
    \item Due lenti convergenti, di distanze focali $f_1 = 30 cm$ e $f_2 = 15 cm$, sono
    disposte in aria, perpendicolarmente all’asse $x$, ad una distanza di $d =
    20 cm$. Anche i fuochi delle lenti sono situati sull’asse $x$. Un oggetto
    viene posto ad una distanza $p = 10 cm$ dalla prima lente.

    \begin{enumerate}
        
        % 4.1
        \item Determinare l’ingrandimento complessivo e la posizione
        dell’immagine in uscita alla seconda lente
        \newline 
        \textbf{Risoluzione}:

        % 4.2
        \item Tracciare graficamente il percorso ottico e la posizione finale dell’oggetto
        \newline 
        \textbf{Risoluzione}:

    \end{enumerate}

\end{enumerate}

\end{document}