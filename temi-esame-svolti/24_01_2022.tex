
\documentclass{article}
\title{Risoluzione tema esame di fisica 2 del 24 gennaio 2022}
\author{Lancillotto dal lago}
\date{\today}

\usepackage[utf8]{inputenc}
\usepackage{enumerate}
\usepackage{amsmath}
\usepackage{mathtools}
\usepackage{graphicx}

\usepackage[right=1.5cm, left=1.5cm]{geometry}

\graphicspath{ {./} }



\begin{document}
\pagenumbering{gobble}
\maketitle

\begin{enumerate}
    \item Una sfera cava di polistirolo ($\epsilon_R = 2.6$) ha raggio esterno $r_2 = 20cm$, mentre la cavità interna ha raggio $r_1=10cm$.
    Su di essa è posta una distribuzione di carica uniforme $\rho=10^{-9} C/m^3$. Calcolare:
    \begin{enumerate}

        \item Il modulo del campo elettrico radiale $E(r)$ al centro della sfera in ogni punto dello spazio
        \newline
        \textbf{Risoluzione}: Si calcola la carica del polistirolo con
        \begin{align}
            v & = \frac{4}{3} \pi r_{2}^{3} - \frac{4}{3} \pi r_{1}^{3} = \frac{4}{3} \pi (r_{2}^{3} - r_{1}^{3}) \\
            q & = v \rho = 1.25 \cdot 10^{-1} m^3
        \end{align} 
        \begin{align*}
            \begin{cases}
                d \leq r_1 \\
                r_1 < d < r_2 \\
                d \geq r_2
            \end{cases}
            \begin{cases}
                0 \\
                \epsilon_0 \epsilon_r \rho r \\
                E = \frac{1}{4 \pi \epsilon_0} \frac{q}{d^2}
            \end{cases}
        \end{align*}

        \item La differenza di potenziale $\Delta V$ fra la superficie interna ed esterna della sfera.
        \newline
        \textbf{Risoluzione}:

    \end{enumerate}
    \item Quattro particelle con la stessa carica $q = -10^{-9} C$ si trovano ai vertici di un quadrato di lato $l = 12 cm$. Si calcoli:
    \begin{enumerate}
        \item Il modulo $E$ dell’intensità del campo elettrico nel centro $O$ del quadrato.
        \newline
        \textbf{Risoluzione}: Nel centro essendo equidistanti ed equipotenti le cariche si annullano a due a due, così che $E(O) = 0C$.
        \item Il modulo $E$ dell’intensità del campo elettrico nel punto medio $M$ di un lato.
        \newline
        \textbf{Risoluzione}: Le componenti orizzontali si annullano a due a due, quelle verticali si sommano
        \begin{align*}
            d_1 = d_2 = \sqrt{l^2 + \left( \frac{l}{2}\right)^2} \\
            d_3 = d_4 = l \\
        \end{align*}
        \begin{align*}
            E & = 2 \cdot \frac{1}{4 \pi \epsilon_0} \frac{q}{d_{1}^{2}} \frac{l}{\sqrt{l^2 + \left( \frac{l}{2}\right)^2} } + 2 \cdot \frac{1}{4 \pi \epsilon_0} \frac{q}{l^{2}} \\
              & = \frac{q}{2 \pi \epsilon_0} \left( \frac{l}{\left( l^2 + \left( \frac{l}{2} \right)^2 \right)^\frac{3}{2}} + \frac{1}{l^2} \right) = 
        \end{align*}
        \item la d.d.p. tra $O$ ed $M$
        \newline
        \textbf{Risoluzione}: Sapendo che $V = \frac{1}{4 \pi \epsilon_0} \frac{q}{d}$ e che $E(O) = 0$
        \begin{align*}
            V(M) = \frac{q}{2 \pi \epsilon_0} \left( \frac{l}{ l^2 + \left( \frac{l}{2} \right)^2 } - \frac{1}{l} \right) = 
        \end{align*} 

    \end{enumerate}

\end{enumerate}

\end{document}